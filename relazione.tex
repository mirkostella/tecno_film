\documentclass[a4paper]{article}
\usepackage[T1]{fontenc}
\usepackage[utf8]{inputenc}
\usepackage[italian]{babel}
\usepackage{hyperref}
\hypersetup{hidelinks}
\author{Mirko Stella 1201184 \and Sonia Franco 1224437 \and Ruth Genevieve Bousapnamene 1192088 \and Elton Ibra 1234931}
\title{\Huge \textbf {Tecno Film}}
\date{20 Agosto 2022}
\begin{document}
\maketitle
\textbf{Indirizzo del sito}:

Segnaposto indirizzo

\textbf{Login utente}:

\textit{email}:user 

\textit{password}:user
 
\textbf{Login amministratore}:

\textit{email}:admin 

\textit{password}:admin

\textbf{Indirizzo mail di riferimento}:

mirko.stella@studenti.unipd.it

\tableofcontents
\newpage
\section{Introduzione}
Tecno Film \`e un sito che si occupa di esporre in modo organizzato i film presenti al suo interno ed offre la possibililt\`a di noleggiarli o acquistarli agli utenti registrati.

L'idea \`e nata da un gruppo di studenti dell'Universit\`a di Padova appassionati di film che ha voluto usare questa 
passione per fare il progetto di Tecnologie Web del terzo anno del corso di Informatica.

Viene offerta un'area dedicata agli amministratori separata dal resto del sito in quanto specifica per l'amministrazione.
    \section{Tipologie di utenti}
    Tecno Film \`e rivolto ad utenti maggiorenni data la presenza di contenuti che potrebbero non essere adatti a tutte le et\`a per il linguaggio
    o per le scene che potrebbero presentarsi durante la visione di un film.
\`E pertanto consigliata la presenza di un adulto durante la visione di determinati film.   
\subsection{Utente generico}
\textbf{Tra gli utenti generici si considerano gli utenti che non sono registrati o che non sono loggati.}

Un utente generico pu\`o:

    \noindent\textbf{Visualizzare le pagine}:
        \begin{itemize}
            \item Homepage
            \item Classifiche
            \item Regolamento
            \item Chi siamo
            \item Pagine film
        \end{itemize}
    \textbf{Eseguire le operazioni}:
        \begin{itemize}
            \item Registrazione
            \item Login (se registrato)
            \item Ricerca dei film per nome
        \end{itemize}
\subsection{Utente loggato}
    Un account pu\`o trovarsi nello stato Attivo o Bloccato.
    \subsubsection{Loggato con account Attivo}
    Un utente loggato con account Attivo pu\`o:

    \noindent\textbf{Visualizzare le pagine}:
    \begin{itemize}
        \item Tutte le pagine che pu\`o visualizzare un utente generico
        \item I miei film
    \end{itemize}
    \textbf{Eseguire le operazioni}:
    \begin{itemize}
        \item Logout
        \item Acquisto di uno o pi\`u film
        \item Noleggio di uno o pi\`u film
        \item Inserimento di una recensione per film
        \item Eliminazione delle proprie recensioni
        \item Valutazione delle recensioni degli altri utenti come Utili
        \item Segnalazione delle recensioni degli altri utenti
    \end{itemize}

    \subsubsection{Loggato con account Bloccato}
    Un utente loggato con account Bloccato pu\`o:

    \noindent\textbf{Visualizzare le pagine}:
    \begin{itemize}
        \item Tutte le pagine che pu\`o visualizzare un utente con account Attivo
    \end{itemize}
    \textbf{Eseguire le operazioni}:
    \begin{itemize}
        \item Logout
        \item Acquisto di uno o pi\`u film
        \item Noleggio di uno o pi\`u film
    \end{itemize}
\subsection{Amministratore}
    Un amministratore si occupa della gestione dei film e degli utenti e pu\`o tenere sotto controllo gli incassi derivati
    dall'acquisto o dal noleggio dei film.
    Un amministratore loggato pu\`o:

\noindent\textbf{Visualizzare le pagine}:
\begin{itemize}
    \item Riepilogo
    \item Aggiungi film
    \item Segnalazioni utente
\end{itemize}
\textbf{Eseguire le operazioni}:
\begin{itemize}
    \item Logout
    \item Aggiunta di nuovi film 
    \item Eliminazione delle recensioni degli utenti
    \item Modifica dello stato degli utenti
\end{itemize}
\section{Struttura delle pagine}
\subsection{HTML}
La struttura delle pagine \`e stata scritta in HTML5 mantenendo la sintassi XHTML in modo che le pagine possano risultare valide anche per i browser che non
lo supportano.
Viene utilizzato il modello a tre pannelli composto da header,menu e contenuto in modo da garantire all'utente una modalit\`a di navigazione
gi\`a conosciuta.

La validit\`a di ciascuna pagina \`e stata testata utilizzando il codice HTML generato dagli script PHP tramite il tool \href{https://validator.w3.org/}{Markup Validation Service} fornito da W3C.

Nella cartella componenti si trovano i file .html che vengono usati per comporre le parti comuni tra pi\`u pagine e quelle
che si ripetono pi\`u volte all'interno della stessa pagina ma richiedono una personalizzazione in base al loro contenuto.
\subsubsection{Utente}
Le pagine utente contengono un segnaposto \%base\% seguito dal contenuto della pagina.Per ogni pagina il segnaposto \%base\% viene sostituito 
dal corrispondente script php con il codice presente nel file ../componenti/base.html.
I segnaposto relativi alle keywords,descrizione e titolo di ogni pagina vengono anch'essi sostituiti tramite php.
\subsubsection{Amministratore}
La pagina di login non contiene il segnaposto \%base\% perch\`e non \`e presente il men\`u. Tutte le altre pagine utilizzano il codice presente
in ../componenti/base\_admin.html per sostituire il segnaposto \%base\%.

\subsection{CSS}
Viene utilizzato CSS3 che permette l'utilizzo di flex-box per gestire gli elementi.
Flex-box viene utilizzato per disporre gli elementi che costituiscono l'header e per disporre il menu e il contenuto delle pagine.
\`E stato usato anche per disporre le card dei film e per i dettagli contenuti al loro interno.

In tutto il file .css dove possibile sono state usate unit\`a di misura relative dato che non \`e possibile avere un perfetto controllo dell'
output utilizzando unit\`a di misura assolute.

Per rendere il sito responsive e quindi adattarsi a schermi di dimensioni differenti sono stati introdotti dei breakpoint
facendo uso delle media query.
I breakpoint stabiliti sono: 75em,60em,35em.

Le regole applicate fino a 75em rendono raggiungibile il menu (situato a fine pagina) tramite un link all'interno dell'header.
Adottando questo metodo non sono state necessarie regole per rendere visibile il contenuto di un possibile menu ad hamburger.
\textbf{Stampa}
Per la quantit\`a limitata di pagine il file .css comprende anche le regole di stampa(media print).
Queste regole hanno lo scopo di eliminare gli elementi interattivi (ad esempio i pulsanti,caselle di ricerca,menu) e di minimizzare l'uso 
del colore (ad esempio non vengono stampate le immagini di copertina).
Per quanto riguarda le interruzioni di pagina sono state inserite delle regole che impediscono la separazione delle informazioni dei film tra una 
pagina e l'altra.
Per le pagine di login e di registrazione non sono state previste regole di stampa.
\subsection{PHP}
Il rendering delle pagine viene fatto lato server sostituendo i segnaposto presenti nelle pagine HTML
con il contenuto opportuno.

I segnaposto hanno la forma \%segnaposto\%.

Per gestire la connessione al database \`e stato creato il file connessione.php.
Il file contiene le funzioni per aprire e chiudere la connessione e i metodi che servono per interrogare il database.
La connessione viene aperta una sola volta all'inizio dello script corrispondente alla pagina del sito e chiusa alla fine quando non \`e pi\`u necessaria.
Alle funzioni che vengono chiamate dallo script e che richiedono una connessione (che deve essere aperta),questa viene passata come parametro.

Per gestire la variabile superglobale \$\_SESSION \`e stato creato il file sessione.php.
La variabile \$\_SESSION viene usata per fare i controlli sul tipo e sullo stato (loggato o non loggato) di un utente e per tenere traccia del suo ID.
Il sito non prevede la possibilit\`a di essere loggati sia come utente che come amministratore contemporaneamente.

\textbf{Form}

Sono stati previsti dei controlli sui dati inseriti dall'utente tramite le form e in caso di input non validi vengono
mostrati i relativi messaggi di errore.
Inoltre \`e prima di eseguire le query viene fatta la pulizia dell'input che consiste nell'eliminazione degli spazi vuoti
prima e dopo l'input e nella sostituzione di possibili caratteri html in entit\`a html.

\subsection{JS}
Javascript viene utilizzato per controllare i campi delle form,segnalare gli errori ed assistere l'utente durante la fase di inserimento dei dati.
Viene utilizzato lato client per gestire l'invio dei dati delle form in modo da limitare le interrogazioni al server.
Infatti se una form contiene dei valori non validi la richiesta al server viene bloccata.
Nel caso in cui Javascript sia disabilitato le richieste al server vengono inoltrate in ogni caso e i messaggi di errore vengono visualizzati nella pagina di risposta.

Le pagine in cui sono stati inseriti gli script sono:

\noindent \textbf{Utente:}
\begin{itemize}
    \item login.html
    \item registrazione.html
\end{itemize}
\textbf{Amministratore:}
\begin{itemize}
    \item ins\_film\_admin.html
\end{itemize}



\section{Accessibilit\`a}
\section{Database}
\section{Note}

\end{document}