\documentclass[a4paper]{article}
\usepackage[T1]{fontenc}
\usepackage[utf8]{inputenc}
\usepackage[italian]{babel}
\usepackage{hyperref}
\hypersetup{hidelinks}
\title{\Huge \textbf {TecnoFilm}}
\date{}
\begin{document}
\maketitle

\vfill
\paragraph*{Componenti del gruppo:}
\begin{itemize}
    \item Mirko Stella - 1201184 
    \item Sonia Franco - 1224437
    \item Ruth Genevieve Bousapnamene - 1192088  
    \item Elton Ibra - 1234931
\end{itemize}

\paragraph*{Indirizzo del sito:}
Segnaposto indirizzo

\paragraph*{Login utente:}
\begin{enumerate}
    \item [] \textit{email}: user 
    \item [] \textit{password}: user
\end{enumerate}
 
\paragraph*{Login amministratore:}
\begin{enumerate}
    \item [] \textit{email}: admin 
    \item [] \textit{password}: admin
\end{enumerate}

\paragraph*{Indirizzo mail di riferimento:}

mirko.stella@studenti.unipd.it

\vfill

\newpage

\tableofcontents

\newpage
\section{Introduzione}
TecnoFilm è un sito web ideato per offire ai suoi utenti un servizio di acquisto e noleggio di film.
L'idea è nata da un gruppo di quattro studenti del corso di laurea in Informatica dell'università di Padova per la realizzazione di un progetto didattico del corso Tecnologie Web.
La passione per i film e la passione per l'informatica hanno portato alla realizzazione di TecnoFilm.

\section{Tipologie di utenti}
Data la possibilità di contenuti non adatti ai minori durante la visione di alcune pellicole, TecnoFilm permette l'iscrizione esclusivamente a persone maggiorenni.
\subsection{Utente generico}
\textbf{Tra gli utenti generici si considerano gli utenti che non sono registrati o che non sono loggati.}

\noindent Un utente generico può:
    \paragraph*{Visualizzare le pagine:}
        \begin{itemize}
            \item[-] Homepage
            \item[-] Classifiche
            \item[-] Regolamento
            \item[-] Chi siamo
            \item[-] Pagine film
        \end{itemize}
    \paragraph*{Eseguire le operazioni:}
        \begin{itemize}
            \item[-] Registrazione
            \item[-] Login (se già registrato)
            \item[-] Ricerca dei film per nome
        \end{itemize}

\subsection{Utente loggato}
    Un account può trovarsi nello stato Attivo o Bloccato.
    
    \subsubsection{Loggato con account Bloccato}
    Un utente loggato con account Bloccato può:
    
    \paragraph*{Visualizzare le pagine:}
    \begin{itemize}
        \item[-] Tutte le pagine che può visualizzare un utente generico
        \item[-] I miei film
    \end{itemize}

    \paragraph*{Eseguire le operazioni:}
    \begin{itemize}
        \item[-] Logout
        \item[-] Ricerca del film per nome
        \item[-] Acquisto di uno o più film
        \item[-] Noleggio di uno o più film
    \end{itemize}

    \subsubsection{Loggato con account Attivo}
    Un utente loggato con account Attivo può:

    \paragraph*{Visualizzare le pagine:}
    \begin{itemize}
        \item[-] Tutte le pagine che può visualizzare un utente loggato con account Bloccato
    \end{itemize}

    \paragraph*{Eseguire le operazioni:}
    \begin{itemize}
        \item[-] Tutte le operazioni che può eseguire un utente loggato con account Bloccato
        \item[-] Inserimento di una recensione per film
        \item[-] Eliminazione delle proprie recensioni
        \item[-] Valutazione delle recensioni degli altri utenti
        \item[-] Segnalazione delle recensioni degli altri utenti
    \end{itemize}

\subsection{Amministratore}
Un amministratore dispone di una form di login diversa dagli utenti.
Un amministratore si occupa della gestione dei film e e degli utenti.
Un amministratore che ha effettuato correttemente l'accesso può:

\paragraph*{Visualizzare le pagine:}
\begin{itemize}
    \item[-] Riepilogo
    \item[-] Aggiungi film
    \item[-] Segnalazioni utente
\end{itemize}

\paragraph*{Eseguire le operazioni:}
\begin{itemize}
    \item[-] Logout
    \item[-] Aggiunta di nuovi film 
    \item[-] Eliminazione delle recensioni degli utenti
    \item[-] Modifica dello stato degli utenti
\end{itemize}

\section{SEO}
Per avere migliori risultati con SEO, innanzitutto si è eseguito un controllo sulla validità del codice HTML e CSS.\\
Si sono considerati due tipi di ricerca più probabili:
\begin{itemize}
    \item [-] L'utente conosce il nome del sito e vuole accedervi
    \item [-] L'utente conosce il titolo del film che vuole acquistare o noleggiare
\end{itemize}
Per il primo tipo di ricerca il nome del sito è posto come logo in ogni pagina tramite un tag <h1>, oltre che nel tag <title> e nelle keywords di tutte le pagine.

Per il secondo tipo di ricerca, il titolo di un film è presente nella propria pagina film sia nel tag <title> che nelle keywords, ma soprattutto con un tag <h1> all'inizio del contenuto.

Per le keywords sono state utilizzate altre parole che rimandano ai contenuti e allo scopo dell pagine, come per esempio: Acquisto, Noleggio, Nuove Uscite.

\section{Struttura}

\subsection{Pagine Utente}

\subsubsection{Header}

Nell'header sono presenti:
\begin{itemize}
    \item [-] Logo: è identificato tramite un tag \textit{<h1>}
    \item [-] Barra di ricerca: consente all'utente di ricercare i film per nome
    \item [-] Link per la Registrazione
    \item [-] Link per la login/logout
\end{itemize}

\subsubsection{Breadcrumb}
Per evitare il disorientamento dell'utente e aiutarlo nella navigazione, è stata aggiunta una breadcrumb per ogni pagina tramite il tag semantico \textit{<nav>}.

\subsubsection{Menù}
Il menù è identificato anch'esso dal tag semantico \textit{<nav
>} e le opzioni sono elencate come \textit{unordered list}.

Il sito non presenta un'elevata ampiezza. Il menù è composto da:

\begin{itemize}
    \item[-] Home
    \item[-] Classifiche
    \item[-] Regolamento
    \item[-] Chi siamo
    \item[-] I miei film (questa opzione compare solo se si è loggati) 
\end{itemize}

Per evitare link circolari, se ci si trova in una delle pagine del menù, il link viene disattivato e reso riconoscibile dagli altri.

\subsubsection{Contenuto}

Com'è organizzato il contenuto per ogni pagina?

\begin{itemize}
    \item \textbf{Home}:
    
    Nella Home viene proposta una panoramica della raccolta di film presente nel sito.

    Vengono mostrate le categorie \textit{Nuove Uscite}, \textit{Scelti per te} (questa categoria solo se si è loggati), e una categoria per genere.
    Per ogni categoria vengono elencati cinque film. Per vedere altri film di quella categoria è possibile premere il pulsante \textit{Vedi altro}.

    Premento sul titolo o sulla copertina di un film si viene reindirizzati alla pagina di quel film.

    \item \textbf{Classifiche}
    
    Strutturata allo stesso modo della pagina \textit{Home}, cambiano le categorie.

    Le categorie sono: \textit{Top 5 della settimana}, \textit{Top 10 più visti}, \textit{Top 10 più votati}.

    Anche in questo caso, premento sul titolo o sulla copertina di un film si viene reindirizzati alla pagina di quel film.

    \item \textbf{I miei film}
    
    La pagina è disponibile solo per gli utenti loggati.

    La pagina offre all'utente un riepilogo dei film acquistati, e dei film noleggiati. Nel caso dei film noleggiati, si fa una suddivisione tra i noleggi scaduti e non.

    \item \textbf{Regolamento}

    Come si può intuire dal nome, è un reminder delle condizioni di acquisto e di noleggio, e del comportamento a cui bisogna attenersi.

    \item \textbf{Chi siamo}
    
    Si tratta di una breve presentazione.

    \item \textbf{Pagina film}
    
    Esiste una pagina di questo tipo per ogni film presente nel database.

    Si tratta di una pagina con tutte le informazioni del film: titolo, copertina, data di uscita, durata, genere, trama.

    In questa pagina, gli utenti loggati e attivi possono lasciare una recensione o valutare utile/segnalare quelle degli altri utenti.
    Si è prevista la possibilità di lasciare una sola recensione per film.

    Da questa pagina è possibile acquistare o noleggiare il film.
    
    \item \textbf{Registrazione} e \textbf{Login}
    
    La pagina di registrazione permette di iscriversi alla community. La pagina di login permette di accedervi.

\end{itemize}

Sono presenti alte pagine di servizio:

\begin{itemize}
    \item \textbf{Conferma Acquisto/Noleggio}

    A questa pagina ci si può arrivare solo dopo aver fatto richiesta di acquisto o noleggio nella pagina del film.
    Viene richiesta un'ulteriore conferma.

    \item \textbf{Search Results}

    Se si effettua una ricerca tramite la barra di ricerca dell'header, si viene reindirizzati a questa pagina.
    Se vengono trovati riultati, questi verranno motrati come un elenco.

\end{itemize}

\subsection{Pagine Amministratore}

\subsubsection{Header}

Nell'header delle pagine amministratore sono presenti:
\begin{itemize}
    \item [-] Logo: è identificato tramite un tag \textit{<h1>}
    \item [-] Link per la Registrazione
    \item [-] Link per la login/logout
\end{itemize}

\subsubsection{Breadcrumb}
Anche per le pagine amministratore è stata aggiunta una breadcrumb per ogni pagina tramite il tag semantico \textit{<nav>}.

\subsubsection{Menù}
Anche in questo caso il menù è identificato anch'esso dal tag semantico \textit{<nav>} e le opzioni sono elencate come \textit{unordered list}.

Il menù dell'amministratore è composto da:

\begin{itemize}
    \item[-] Riepilogo
    \item[-] Aggiungi film
\end{itemize}

Anche in questo caso, per evitare link circolari, se ci si trova in una delle pagine del menù, il link viene disattivato e reso riconoscibile dagli altri.

\subsubsection{Contenuto}

Com'è organizzato il contenuto delle pagine amministratore?

\begin{itemize}
    \item \textbf{Riepilogo}
    
    É una pagina che funge da riepilogo per l'amminitratore.
    Sono presenti due tabelle: una tabella per gli utenti e una per i film.

    Tramite la tabella degli utenti è possibile avere una panoramica delle segnalazioni, e dello stato.
    
    Tramite la tabella dei film è possibile vedere gli incassi totali, e gli incassi parziali per film.

    \item \textbf{Aggiungi film}
    
    É la pagina tramite cui un amministratore può inserire un nuovo film nel db. I controlli sulla form vengono eseguiti sia tramite \textit{php} che tramite \textit{javascript}.

    \item \textbf{Segnalazioni utente}
    
    Esiste una pagina di questo tipo per ogni utente registrato.
    
    Si tratta di un riepilogo delle informazioni dell'utente e di un elenco delle recensioni effettuate.

    Tramite questa pagina l'amministratore può cambiarne lo stato da Attivo a Bloccato, o viceversa, e eliminarne le recensioni ritenute inappropriate.

    \item \textbf{Login}
    
    Permette all'amministratore di loggarsi.
    
\end{itemize}


\section{HTML}
La struttura delle pagine è stata scritta in HTML5, rispettando la sintassi XML.

\paragraph{}Per evitare di ripetere linee di codice, nella cartella \textit{componenti} si trovano i file \textit{.html} il cui codice o si ripete all'interno di una stessa pagina, o è comune a più pagine.

\noindent Tra questi file, i più importanti sono \textit{base.html} e \textit{base\_admin.html}.

\noindent Infatti, essendo la struttura delle pagine la stessa, come si può intuire dal nome si è scelto di utilizzare questi due file come "base": uno per le pagine utente e uno per le pagine admin.

\noindent I due file contengono la sezione \textit{<head>}, e parte del tag \textit{<body>}: il codice relativo all'header, alla barra di ricerca, alla breadcrumb, e al menù.

\noindent Come conseguenza di questa scelta, si è reso necessario l'utilizzo di segnaposti per i \textit{meta-name} \textit{description} e \textit{keywords}, per il tag \textit{<title>}, per i sueggerimenti della barra di ricerca, per la breadcrumb e per il menù.

\paragraph*{}Tutti i segnaposto sono scritti come \textbf{\%segnaposto\%} e vengono sostituiti tramite \textit{php}.

\paragraph*{}Tutte le pagine \textit{.html} sono quindi strutturate così:
\begin{itemize}
    \item [-] segnaposto \%base\%
    \item [-] contenuto della pagina
\end{itemize}

\paragraph*{Nota:}L'unica pagina che non utilizza questa tecnica è \textit{login\_admin.html} perchè non è previsto il menù.

\paragraph{}Il segnaposto \%base\% verrà poi sostituito con il codice del file \textit{base.html} per le pagine utente e con il codice del file \textit{base\_admin.html} per le pagine admin.

\paragraph{}La validità di ciascuna pagina è stata testata tramite il tool \href{https://validator.w3.org/}{Markup Validation Service}, fornito da W3C, utilizzando il codice HTML generato dagli script PHP.

\section{CSS}
Per il codice CSS è stato utilizzato \textit{CSS3}.

\subsection{Utilizzo di Flexbox}
Avendo scelto di utilizzare \textit{CSS3} è stato possibile utilizzare Flexbox.
In particolare, il layout Flex è stato utilizzato per disporre gli elementi dell'\textit{header} e per disporre il menù e il contenuto delle pagine. 
Inoltre, nelle pagine che richiedevano un elenco dei film, è stato sfruttato per disporre le card.
Le card stesse, per distribuire gli elementi al loro interno, ne fanno uso.

\subsection{Media query}
Per rendere il sito responsive e quindi adattabile a schermi di dimensioni differenti sono stati introdotti dei breakpoint
facendo uso delle media query.
I breakpoint stabiliti sono: 75em, 60em, 35em.
Le regole applicate a schermi fino a 75em rendono raggiungibile il menù (situato a fine pagina) tramite un link all'interno dell'header.
Adottando questo metodo non sono state necessarie regole per rendere visibile il contenuto di un possibile menu ad hamburger.

Data la quantià limitata di pagine, le media query sono state utilizzate anche per le regole per la stampa tramite \textit{@media print}.

\subsection{CSS per la stampa}
Per la stampa sono stati eliminati tutti gli elementi ritenuti superflui ai fini della stampa, come pulsanti, barra di ricerca, form per la recensione e menù.

Inoltre, per minimizzare l'uso del colore e dello spazio, per la stampa si è scelto di eliminare le immagini di copertina dei film, e lasciarle esclusivamente nella pagina del film.

Per evitare la separazione di contenuti, sono state introdotte delle regole per le interruzioni di pagina.

Considerando superfluo e poco sensato l'utilizzo della stampa per le pagine di login e registrazione, per queste due pagine non sono state previste delle regole.

\section{PHP}
Il rendering delle pagine viene fatto lato server sostituendo i segnaposto presenti nelle pagine HTML
con il contenuto opportuno.

\subsection{Connessione al DB}
Il file \textit{connessione.php} è il file che contiene le funzioni per la gestione della connessione al DB.
In questo file, infatti, si trovano le funzioni che aprono e chiudono la connessione e quelle che interrogano il DB.

Si è scelto di aprire la connessione all'inizio degli script e chiuderla quando non è più necessaria.

Se all'interno dello script, vengono chiamate alcune funzioni che richiedono la connessione aperta, questa viene passata come parametro. 

\subsection{\textit{\$\_SESSION}}
Il file \textit{sessione.php} è stato creato per personalizzare e gestire le sessioni tramite la variabile superglobale \textit{\$\_SESSION}.

Questa variabile viene utilizzata per eseguire dei controlli sullo stato loggato/non loggato e sul tipo di account che sta navigando (amministratore o utente).

Il sito, infatti, non prevede la possibilità di essere loggati contemporaneamente come sia come utente che come amministratore.
Quando la sessione è attiva come utente, e si effettua la login come amministratore, la sessione utente viene terminata. Questo viene fatto anche a situazione inversa.

\subsection{Controllo sulle form}

I controlli sulle form vengono effettuati sia tramite \textit{php} che tramite \textit{javascript}.

I controlli sugli input effettuati dagli script \textit{php}, mostrano eventuali messaggi d'errore. Nel caso non ci siano errori, vengono mostrati dei messaggi per informare l'utente circa il successo delle operazioni.

Per una questione di sicurezza, prima di memorizzare dati inseriti dall'utente nel db, gli script php effettuano una pulizia dell'input tramite due funzioni offerte dal linguaggio: \textit{trim} e \textit{htmlentities}: la prima per eliminare spazi vuoti prima e dopo l'input, e la seconda per trasformare i caratteri speciali in entità html.

\section{JS}
\textit{Javascript} viene utilizzato per effettuare i controlli sui campi delle form, segnalare gli errori ed assistere l'utente durante la fase di inserimento dei dati.
Viene utilizzato lato client per gestire l'invio dei dati delle form in modo da limitare le interrogazioni al server: se una form contiene dei valori non validi, infatti, la richiesta al server viene bloccata.

Nel caso in cui \textit{Javascript} sia disabilitato le richieste al server vengono inoltrate in ogni caso e i messaggi di errore vengono visualizzati nella pagina di risposta.

Le pagine in cui sono stati inseriti gli script sono:

\paragraph{Utente:}
\begin{itemize}
    \item[-] \textit{login.html}
    \item[-] \textit{registrazione.html}
\end{itemize}

\paragraph{Amministratore:}
\begin{itemize}
    \item[-]  \textit{ins\_film\_admin.html}
\end{itemize}



\section{Accessibilità}

\subsection{Assenza di \textit{CSS}}
In assenza di \textit{CSS}, il sito mantiene una struttura e un aspetto comprensibile.

\subsection{Assenza di \textit{JS}}
In assenza di \textit{Javascript} tutti i controlli vengono comunque eseguiti tramite \textit{php}.ù
Gli elementi che richiedono \textit{Javascript}, come ad esempio il pulsante \textit{Mostra password}, in sua assenza vengono nascosti.

\subsection{Lingua}

Il sito ha come lingua principale l'italiano. Per le parole che hanno lingua inglese, sono stati aggiunti gli opportuni tag \textit{<span>} con le indicazione della lingua usata.

Quando è stato possibile, l'aggiunta degli \textit{<span>} sulla lingua sono stati aggiunti direttamente nel file \textit{html}. Questo però non è sempre stato possibile. Ad esempio, per titoli dei film, per nomi o per parole appartenenti alla trama, o semplicemente per alcuni generi dei film, essendo presi dal db ed essendo quindi elementi dinamici.

In quest'ultimo caso, è stata adottata la seguente tecnica: all'inserimento dei film tramite il form, le parole in inglese devono essere racchiuse tra due accenti circonflessi, così:

\textbf{\^{}}PAROLA\textbf{\^{}}

I due \textbf{\^{}}, tramite un'apposita funzione, vengono poi sostituiti tramite \textit{php} con lo \textit{span}.

\subsection{Attributi ARIA}

\begin{itemize}
    \item [-] \textit{\textbf{aria-label}}
    
    L'attributo \textit{aria-label} è stato utilizzato per la breadcrumb e per il menù.
    Inoltre è stato utilizzato per gli elementi interattivi come input o button, per fornire informazioni ulteriori.

    \item [-] \textit{\textbf{aria-hidden}}
    
    L'attributo \textit{aria-hidden="true"} è stato utilizzato per nascondere elementi che non sono utili a utenti che utilizzano gli screen reader.

    In particolare, la valutazione dei film tramite le stelle è stata nascosta per queste categorie di utenti, ed è stata aggiunto uno \textit{span} con la valutazione, che viene nascosto agli altri utenti tramite \textit{css} con la classe "aiuti".

    \textit{aria-hidden="true"} viene utilizzato anche per "saltare" l'immagine di copertina dei film.

    \item [-] \textit{\textbf{role}}
    
    L'attributo aria \textit{role} è stato usato per identificare il contenuto principale delle pagine tramite \textit{role="main"}, e per identificare la barra di ricerca tramite \textit{role="searchbox"}.
    Avendo utilizzato il tag semantico \textit{<nav>} per la breadcrumb e per il menù si è reso superfluo l'uso dell'attributo \textit{role="navigation"}.

\end{itemize}

\subsection{Tabindex e accesskey}

\paragraph*{Tabindex:}
data la struttura del sito, non è sembrato necessario modificare l’ordine dei tabindex manualmente.
É stato utilizzato esclusivamente per rendere non raggiungibile tramite la navigazione sequenziale da tastiera le copertine dei film.

\paragraph*{Acccesskey:}
si è inoltre fatto uso di accesskey. In questo caso, si è cercato di impostare come accesskey l'iniziale dell'elemento a cui fa riferimento.

\subsection{Contrasti}

Per l'utilizzo dei colori si è prestata attenzione ai contrasti minimi per il rispetto degli standard del WCAG.

\subsection{Tabelle}
Per le due tabelle presenti nella sezione amministratore, oltre alla corretta suddivisione in \textit{thead}, \textit{tbody} e \textit{tfoot} si è fatto uso dell'attributo \textit{summary} per fornire una descrizione della tabella a cui fa riferimento, del tag \textit{<caption>}, dei tag \textit{<th>} per le intestazioni e dell'attributo \textit{scope}.

\section{Test}

\section{Suddivisone del lavoro}

\begin{itemize}
    \item [] \textbf{Stella Mirko}
    \begin{itemize}
        \item codice \textit{html}
        \item script \textit{php}
        \item script \textit{javascript}
        \item relazione
        \item css
    \end{itemize}
    \item [] \textbf{Franco Sonia}
    \begin{itemize}
        \item codice \textit{html}
        \item script \textit{php}
        \item relazione
        \item css
        \item css per la stampa
        \item riempimento db
    \end{itemize}
    \item [] \textbf{Bousapnamene Ruth Genevieve}
    \begin{itemize}
        \item tabelle pagina \textit{Riepilogo} sezione amministratore
    \end{itemize}
    \item [] \textbf{Ibra Elton}
    \begin{itemize}
        \item creazione tabelle db
    \end{itemize}
\end{itemize}

\end{document}